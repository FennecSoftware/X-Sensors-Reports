\documentclass[12pt,a4paper]{article}
\usepackage[a4paper,left=2cm,right=2cm,top=3cm,bottom=3cm]{geometry}
\usepackage[utf8]{inputenc}
\usepackage[T1]{fontenc}
\usepackage{amsmath}
\usepackage{amssymb}
\usepackage{graphicx}
\usepackage{float}
\usepackage{nicefrac}
\usepackage[ngerman]{babel}

\usepackage[obeyspaces]{url}

\newcommand{\ue}{$\mu\varepsilon$}



\title{Genauigkeitsanalyse Messequipment\\[3ex] \small{Analyse für Kalibration X-106 30\ue}}

\author{Mirco Huber}



\begin{document}
	\maketitle
	\newpage
	\section{Bestandsaufnahme / Ausgangslage}
	Im Folgenden wird untersucht, ob das aktuell vorhandene Messequipment hinreichend genau ist, um den X-106-Sensor mit eimen Fullscale von 30\ue zu kalibrieren. Aus den Kalibrationsparametern, welche in NAV hinterlegt sind, gehen folgende Anforderungen/Eigenschaften hervor:
	\begin{table}[H]
		\centering
		\begin{tabular}{l|l}
			Eigenschaft & Wert \\
			\hline
			Messbereich & $\pm 30$\ue\\
			Nullpunkt & 0.5V\\
			Signalhub & 9.5V\\
			Nichtlinearität & $\le 0.3$\%FS\\ 
			Hysterese & $\le 0.3$\%FS\\ 
		\end{tabular}	
	\end{table}\noindent
	Der verkettete Fehler aus Nichtlinearität und Hysterese muss folglich $\le (0.3+0.3) \%FS = 0.6\%FS$ sein. Für die folgenden Betrachtungen wird daher ein zulässiger summierter Fehler von max $0.5\%FS$ angenommen. Somit ist etwas Marge vorhanden, zudem ist es unwahrscheinlich, dass die Stelle mit dem grössten Hysteresefehler mit jener der grössten Linearitätsabweichung zusammenfällt. Absolut betrachtet bedeutet ein zulässiger Fehler von 0.5\%FS $0.005*9.5V = 0.0475V$. Damit ein Fehler von einem halben Prozent überhaupt erkannt werden kann, müssen mindestens 200 Messpunkte im Intervall von [0,FS] zu liegen kommen. Aus messtechnischer Sicht sollte das Prüfmittel um ein Faktor 10 genauer sein; somit sollten $\ge 2000$ Messpunkte in diesem Intervall differenzierbar sein.
	\subsection{USB-Messinterface}
	Zum Aufzeichnen von Messkurven wird das hauseigene \textit{USB-Messinterface} (Artikel 100431) eingesetzt. Dieses besteht aus einem 16-Bit AD-Wandler von MCC / Digilent mit einem Messbereich von $\pm 10V$ sowie hauseigenen Messverstärkern mit einer Verstärkung von $\pm 4mV/V \rightarrow \pm 10V$\\
	Gemäss Datenblatt (\path{X:\Produkte\USB_Messinterface\Entw\Docu}) beträgt die integrale Nichtlinearität (INL) $\pm1.8 LSB$, womit von den ursprünglichen 16 Bits nur 14.2 ausgewertet werden.\\
	Rechnerisch ergibt sich aus obigen Spezifikationen eine Messauflösung von
	\begin{equation}
		2^{14.2}/20 \approxeq 941 S/V
	\end{equation}	
	wobei S/V für \textit{Samples per Volt} steht.
\end{document}