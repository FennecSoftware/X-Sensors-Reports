\documentclass[12pt,a4paper]{article}
\usepackage[a4paper,left=2cm,right=2cm,top=3cm,bottom=3cm]{geometry}
\usepackage[utf8]{inputenc}
\usepackage[T1]{fontenc}
\usepackage{amsmath}
\usepackage{amssymb}
\usepackage{graphicx}
\usepackage{float}
\usepackage{nicefrac}
\usepackage[ngerman]{babel}
\title{Untersuchung Rauschverhalten\\[3ex] \small{ X-106 mit Empfindlichkeit 0...30$\mu\varepsilon$; Vergossen}}

\author{Mirco Huber}




\begin{document}
\section*{Details Lieferung der Funktionsmuster September 2023}
Hochsensibler Dehnungssensor mit Messbereich $30\mu\varepsilon$, Living-Zero 0.5V und Hub 9.5V\\\noindent
Modifikationen gegenüber Standard-X-106:
\begin{table}[H]
	\begin{tabular}[t]{ll}
		DMS:		&Hochsensible DMS (303218) mit höherem k-Faktor (GK geklebt 303505)\\[.5em]	
		Print:		&Verstärkung massiv erhöht (R16 $\rightarrow 390k\Omega$ )\\
					&Tiefpassfilter von $f_c=5kHz$ auf $f_c=100Hz$ bzw. $f_c=70Hz$ angepasst\\[.5em]
					& Modifikationen:
					\begin{tabular}[t]{l|l|l}
						$f_c$&R24 & C24\\
						\hline
						 $100Hz$	&7k		&220nF\\
						  $70Hz$	&10k	&220nF\\
						
					\end{tabular}\\[3em]
		Firmware: 	&Project FennecFox, Git Tag V0.0.0\\
					&Event-Driven-Ansatz\\
		Signal		&$0..30\mu\varepsilon \rightarrow 0.5...10V$ aka. $-30...30\mu\varepsilon \leftrightarrow -9...10V$ mit NP 0.5V
	\end{tabular}	
\end{table}

	
\end{document}