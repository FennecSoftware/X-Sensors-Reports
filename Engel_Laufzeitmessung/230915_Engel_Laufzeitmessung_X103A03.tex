\documentclass[10pt,a4paper]{article}
\usepackage[utf8]{inputenc}
\usepackage[a4paper, total={6in, 8in}]{geometry}
\usepackage[T1]{fontenc}
\usepackage{amsmath}
\usepackage{amssymb}
\usepackage{graphicx}
\usepackage{nicefrac}
\usepackage{float}
 \thispagestyle{empty}
\author{Mirco Huber}
\begin{document}
\noindent
Um die Signallaufzeit des X-103 A03 Print zu untersuchen, wurde ein solcher Print mit einem DMS-Simulator verbuden. Anschliessend wurden die DMS-Signale sowie das Ausgangssignal des Prints mit einem Aufzeichnungsgerät mit 30 kHz aufgezeichnet. Mit dem Simulator wurden Sprünge mit $\Delta S = 0.5 \nicefrac{mV}{V}$ generiert. Signale wurden zu vergleichszwecken normiert auf den bereich [0,100]. \\\newline
\noindent Aus der Aufzeichnung lässt sich schließen, dass durch die Signalverarbeitung auf dem Print ein Zeitversatz von 10...30 $\mu s$ entsteht. Der Signalpfad ist rein analog und somit kaum schneller realisierbar. Der Versatz kommt durch Kapazitäten zustande, die jedoch für die Signalkonditionierung (Filterung etc.) benötigt werden. \\\newline
Die untenstehende Abbildung zeigt einen Ausschnitt aus der Laufzeitanalyse. Das DMS-Signal ist etwas verrauscht, da es unverstärkt mit einem 16bit-ADC aufgezeichnet wurde, was aber für die Betrachtung des Zeitverhaltens irrelevant ist. Das vertikale Hauptgitter ist in Schritte von $1 ms$ unterteilt, das Subgitter in $100\mu s$ Schritte.
\noindent Die mechanische Dehnung der Struktur, auf welche die Dehnungssensoren aufgeschraubt werden, wird nahezu in Echtzeit an die DMS-Stellen propagiert. Somit lässt sich schliessen, dass die Signalverarbeitung mit dem X-103-A03-Print.
\begin{itemize}
	\item nahezu in Echtzeit verläuft
	\item nur durch Verzicht auf (Tiefpass-) Filter eventuell noch minim beschleunigt werden kann
	\item die rein analoge Signalverarbeitung in dieser Anwendung die schnellstmögliche ist.
\end{itemize}

\begin{figure}[H]
	\centering
	\includegraphics[width=1\linewidth]{C:/Users/Mirco/Downloads/Engel_Signallaufzeit_X103A03}
	\label{fig:engelsignallaufzeitx103a03}
\end{figure}
\end{document}